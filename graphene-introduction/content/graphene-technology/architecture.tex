The Graphene technology is a framework that is available under MIT license and that can be used to realize a Blockchain that constitutes the following components which are described
individually.

\subsection{ Transactions }
\label{section:architecture:transactions}

When users want to interact with any Blockchain, they construct so called \emph{transactions} and transmit to the network (see \cref{section:architecture:networking}). These present messages that contain instructions about what particular \emph{operation(s)} a user wants to use. A common operation is the simple \emph{transfer} operation that comes with transfer-specific instructions that provides the necessary information for this action, such as the sender, receiver, the amount to transfer as well as an optional encrypted memo. To allow multiple operations to take place subsequently, multiple operations can be bundled into a single transaction.\\
To identify against the system, transactions are cryptographically signed by the users. These signatures \emph{authenticate} a user and provide \emph{authorization} for the operations in the transaction. Depending on the implementation, 
transactions can be required to include a fee or other rate-limiting measure to prevent spamming the network.

\subsection{ Blockchain }

The Blockchain serves as a journal (e.g. a ledger) of user-signed instructions that become a binding agreement as soon as they are included into a block. After inclusion into a block, the agreements are stored indefinitely by means of a hash-linked-list (the Blockchain). From this ordered sequence of transactions, a \emph{current state} (think: account balances) can be determined by processing all transactions consecutively starting at the very first block. As we will see later, the software will ensure that instructions that are stored in the Blockchain have been successfully authenticated and validated. For validating and processing of operations, a common set of rules define the consequences of particular actions, which are part of the of the blockchain protocol (see \cref{section:architecture:protocol}).

\subsection{ Networking }
\label{section:architecture:networking}

A blockchain merely defines a means of storage and can be used in a non-distributed, single-participant fashion as well as in a distributed internet-based mesh network often referred to as Peer-2-Peer (P2P) network. In the latter case, multiple parties are connected with each other in a way that incoming transactions are forwarded to every other connected participant. A transaction ultimately reaches a so called \emph{block producer}. A block producer verifies incoming transactions against a hard-coded protocol and bundles them into a single block that is added to the existing blockchain. At this point, a transaction is considered confirmed and executed. The effects of an executed operation on the current state are defined in the blockchain protocol  (see \cref{section:architecture:protocol}).

\subsection{ Consensus }
Consensus is the process by which a community comes to a universally recognized, unambiguous agreement on a piece of information. In the context of blockchains, consensus means agreement about the validity rules for transactions (i.e. the blockchain protocol - see \cref{section:architecture:protocol}), and the order in which they have been observed by the blockchain. This ultimately results in an agreement about the \emph{current state} that is build deterministically from those validity rules and the sequence of transactions.

The most commonly known consensus scheme is Proof-of-Work (PoW). Most dominant disadvantage is the heavy power consumption and the scalability in terms of transactions per second and confirmation times. The BitShares Blockchain makes use of an algorithm called Delegated Proof of Stake (DPoS) that was developed specifically to replace the wasteful 'mining' process, increase throughput and reduce reaction times of the blockchain. It is a tremendous improvement when it comes to consumption of electricity.

DPoS allows to generate a new block at fixed rate (block production/confirmation time) with minimal computational requirements. This means that the blockchain can process more transactions in significantly less time and at almost no cost when compared to PoW-based Blockchains\footnote{\url{https://steemit.com/dpos/@dantheman/dpos-consensus-algorithm-this-missing-white-paper}}. Block production is performed by a set of so called \emph{witnesses} (block producers) that take turns. After every turn, the order of block producers is randomized in a deterministic manner such that all parties agree on the new order.

\subsection{Protocol}
\label{section:architecture:protocol}

The most essential part of blockchain technologies is here referred to as blockchain protocol. It defines the behavior of the entire system including consequences and side-effects when processing transactions. Users utilize particular features by crafting a transaction that contains a particular letter-of-interest (also referred to as \emph{operation}).

Since the Blockchain, as a storage, only stores incremental changes (e.g. transfers), the final balance of each account together with other information needs to be tracked separately in the \emph{current state}.

It is important to note that the protocol is deterministic in the sense that the very same state is generated when applying the same sequence of operations (as provided by the blockchain). This makes blockchain technologies tamper proof and auditable.

In BitShares, over 50 operations are available (as of early 2018). Each of them hooks into the Blockchain protocol at least three times:
\begin{itemize}
\item \textbf{Validation}: During validation, the raw instructions (also referred to as \emph{payload}) are checked for consistency. E.g., in case of a transfer, we ensure that the amount to transfer is positive.
\item \textbf{Evaluation}: In the evaluation step, the operation-specific instruction is validated against the current state of the blockchain. In case of a transfer, we here ensure that the amount to be transferred is available in the account of the sender.
\item \textbf{Application}: This step takes action in the sense that it modifies the current state. In the case of a transfer, we here reduce the account balance of the sender and increase the account balance of the receiver according to the amount of tokens transferred.
\end{itemize}

\paragraph{Example: Transfer operation}
Consider a simple \emph{transfer} operation that sends funds from one account to another. Here, the protocol defines the validation rules such that negative amounts are prevented. The evaluation ensures that the sender cannot transfer more than what is in his account balance. When applying a transfer from Alice to Bob, Alice is credited the transferred amount while Bob receives the amount. Here, \emph{transfer} refers to the operation \emph{type}, while the sender, receiver, and amount refers to the operation-specific instructions. Obviously, different operation types come with different instructions.

\subsection{Extensibility }

The Graphene technology is extensively modularized and implements its operations independently of each other. This allows for adding new features once the corresponding code, which the implements validation, evaluation and application methods, reaches maturity. In a sense, operations for a Blockchain implemented with the Graphene technology framework are \emph{smart-contracts} and allows for extending the range of functions of the system. This can be done in a \emph{static} or \emph{dynamic} fashion, which will be illustrated with two examples.

\paragraph{Static smart contracts} Example BitShares Blockchain: In contrast to other smart-contracting platforms, the BitShares Blockchain requires new features to be vetted by the core developers and approved by the core token holders before they can be installed by means of a network-wide protocol upgrade. As a consequence the platform is considered much more solid as new features require to go through multiple stages of quality assurance. These protocol upgrades are well coordinated and already happened 28 times (Q3/2018) in the past.

\paragraph{Dynamic smart contracts} Example EOSIO Software: While many key components of EOSIO are heavily reworked, they concepts of the technology remain the same. The EOSIO Software allows that smart contracts are put on the blockchain dynamically and are then interpreted on the fly. This allows to be flexible in the provided smart contracts, but it completely removes the review and quality assurance process and the user is required to trust every single smart contract.

\subsection{Performance and Scalability}
Blockchains utilizing the Graphene framework have publicly demonstrated sustaining over 3,000 (three thousand) \emph{transactions} per second and over 22,000 \emph{operations} per second on a distributed test network, as shown by the stress test of the BitShares Blockchain. This technology can easily scale to over 100,000 (hundred thousand) or more transactions per second with relatively straightforward improvements to server capacity and communication protocols.\\
To achieve this industry-leading performance, the Graphene technology has borrowed lessons learned from the LMAX Exchange\footnote{\url{https://martinfowler.com/articles/lmax.html}}, which is able to process 6 million transactions per second. Among these lessons are the following  key  points:
\begin{itemize}
\item Keep  everything  in  memory.
\item Keep  the  core  business  logic  in  a   single  thread.
\item Keep  cryptographic  operations  (hashes  and  signatures)  out  of  the  core  business  logic.
\item Divide  validation  into  state-dependent  and  state-independent  checks.
\item Use  an  object  oriented  data  model.
\end{itemize}
By following these simple rules, the Graphene technology is theoretically able to process \textgreater 10,000 (ten thousand) transactions per second without any significant effort devoted to optimization. To put things into perspective\footnote{\url{http://blocktivity.info/}}, at peak times, the Ethereum and Bitcoin Blockchain jointly process roughly 0.7\% of the peak capacity of the BitShares Blockchain (Q3/2018, maximum capacity according to distributed stress testing) as an example for a Graphene based blockchain.
